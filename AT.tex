\documentclass[main.tex]{subfiles}
\begin{document}

\section{Algebraic Topology}
There is a lot of specific things to remember for the AT quals, like exicision, Mayer-Vietoris, etc. I will try to write some of them down as I am reviewing them.

\section{Covering space}
The theory of covering spaces is strongly related to the theory of groups. Super high-level wise, this is the Koszul duality between connected pointed spaces and $E_1$ spaces.

Let $(X,x)$ be a pointed space
In this case, here's the fundamental theorem/construction:

\begin{construction}
Given a covering space $(P, p) \rightarrow (X,x)$, then we have an injection of fundamental groups $\pi_1 (P, p) \rightarrow \pi_1 (X,x)$. Inversely, given a subgroup, then then we can construct $(P, p)$ as follows:

Let $(\tilde{X}, \tilde{x})$ be the universal cover of $(X,x)$ (the universal cover can be constructed as homotopic equivalent classes of path), this space as a $\pi_1(X,x)$ action (in fact it is a $\pi_1(X,x)$ fiber bundle over $X$), and simply take the $\pi_1(P, p)$ quotient.
\end{construction}

The dual construction gives the following equivalence of categories:
\begin{theorem}
There is an equivalence of categories between connected pointed covering spaces over $(X, x)$ and the opposite poset category of subgroups of $\pi_1(X,x)$.
\end{theorem}

The unpointed version is more interesting:
\begin{construction}
Given a connected space $(X,x)$, then any covering space $p: P \rightarrow X$ gives an action of $G = \pi_1(X,x)$ on the fibers $p^{-1} x$. Conversely, given any set $S$ with a $G$ action, we can construct a covering space $\tilde(X) \times_G S$. 
\end{construction}

Here's the fundamental theorem:
\begin{theorem}
The construction above gives an equivalence of category between the category of covering spaces and the category of $G$-sets.
\end{theorem}

\begin{remark}
From the equivalence, we can start translating notions on both side:

\begin{enumerate}
    \item A covering space $P$ is connected iff $G$ acts transitively on the fibers.
    \item The group of deck transformations (automorphisms) of a covering space $P$ is (with a choice of basepoint $p$) $N(\pi_1(P,p))/\pi_1(P,p)$, $NH$ is the normalizer of $H$ (in the ambient group $G$.
    \item A connected covering is called normal if the group of deck transformation acts transitively on the fibers. This is equivalent to $\pi_1(P,p)$ being a normal subgroup (easily shown from above).
\end{enumerate}
\end{remark}

Given $X, Y$ spaces, then 
$$
\pi_1(X \vee Y) \cong \pi_1(X) * \pi_1(Y)
$$
where $*$ is the free product (the coproduct in the category of spaces).

We also have the Van Kampen theorem:
\begin{theorem}
Let $X = U_1 \cup U_2$, $U_1, U_2, U = U_1 \cap U_2$ are all connected, take $x \in U$, then we have the pushout diagram of fundamental groups:
\begin{equation}
    \begin{tikzcd}
\pi_1 (U, x) \arrow[r] \arrow[d]
& \pi_1 (U_1, x) \arrow[d] \\
\pi_1(U_2, x) \arrow[r]
& \pi_1(X,x)
\end{tikzcd}
\end{equation}
\end{theorem}

For product it is much easier:
$\pi_n(X \times Y) \cong \pi_n(X) \times \pi_n(Y)$.

\section{Homology, Cohomology}
\subsection{Standard long exact sequences}
\todo[inline]{cofiber sequence, Mayer-Vietoris, excision}

\subsection{Universal Coefficients theorem}


\subsection{Poincare duality}


\subsection{Standard spaces}
\todo[inline]{$S^n, CP^n, RP^n,$ Klein bottle, gluing of genus g curves, their chain complex, homology, and homotopy group of the genus g curves.}

\end{document}