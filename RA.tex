\documentclass[main.tex]{subfiles}
\usepackage[utf8]{inputenc}

\documentclass[main.tex]{subfiles}
\begin{document}

\section{Real Analysis}

Oh man I need a lot a lot of work here. 

\subsection{General Measure Theory}

\subsubsection{$\sigma$ Algebras, Measureable Spaces, and Measures}
The story of measure theory is about trying to generalize the notion of volume or size. 
We are following these notes: \url{https://www.math.ucdavis.edu/~hunter/measure_theory/measure_notes.pdf}
We first start with the notion of outer measure:

\begin{definition}
An outer measure $\mu_^*$ on a set $X$ is a function 

$$
\mu^* : 2^X \rightarrow [0, \infty]
$$ 
such that 
\begin{enumerate}
    \item $\mu^*(\emptyset) = 0$;
    \item if $E \subset F \subset X$, then $\mu^*(E) \leq \mu^*(F)$;
    \item (Countable subadditivity) if $E_i$ are countable collection of subsets of $X$, then 
    $$
    \mu^*(\bigcup_{i + 1} ^{\infty} E_i)\leq \sum_{i = 1} ^{\infty} \mu^*(E_i)
    $$
\end{enumerate}
\end{definition}
Note that $\mu^*$ is not assumed to be additive even if ${E_i}$ are disjoint. In fact, they often are not.

The condition we really want is countable additivity, which is that when $E_i$ are disjoint subset, then the it is additive rather than subadditive (aka it is an equal sign). However, this can't always happen (not true over the real line $\mathbb{R}$). So we instead restrict on the subset that we are looking at. This is where we need $\sigma$-algebras:

\begin{definition}
A $\sigma$-algebra on a set $X$ is a connection $\mathcal{A}$ of subsets of $X$ such that
\begin{enumerate}
    \item $\empty, X \in A$;
    \item if $A \in \mathcal{A}$, then so is $A^c \in \mathcal{A}$. $A^c$ is the complement.
    \item if $A_i \in \mathcal{A}$ for $i \in \mathbb{N}$, then 
    $$
    \bigcup_{i=1} ^{\infty} A_i,  \ \bigcap_{i=1} ^{\infty} A_i
    $$
    are in $\mathcal{A}$
\end{enumerate}
\end{definition}
Those is is a collection of subsets that is contains the empty set, close under taking complements and countable unions.
Examples are $\{ \emptyset, X \}$, and $\mathcal{P}(X) = 2^X$.

Now we can define a measureable space, that is, a setting where we can define measures:

\begin{definition}
A measureable space is a pair $(X, \mathcal{A})$, where $X$ is a set and $\mathcal{A}$ is a $\sigma$-algebra on $X$.
\end{definition}

Any topological space gives a measureable space:

\begin{definition}
Given a topolgical space $X$ with topology $\tau$, then the Borel topology is the smallest measureable space containing all the open sets (thus also close sets) of $X$. 
\end{definition}

Given a measureable space $(X, \mathcal{A})$, we can now define a measure:

\begin{definition}
A measure $\mu$ on $(X, \mathcal{A})$ is a function 
$$
\mu: \mathcal{A} \rightarrow [0, \infty]
$$
such that 
\begin{enumerate}
    \item $\mu(\emptyset) = 0$;
    \item It is countably additive.
\end{enumerate}
\end{definition}

Ofc measures restricts to a measureable subset. 

\begin{example}
Let $X$ be a set, then $\nu: \mathcal{P}(X) \rightarrow [0, \infty]$ by cardinality is a measure, called the counting measure. 
\end{example}

A measure zero set, is a measureable net $N$ such that $\mu(N) = 0$. A property hold for all element away from a measure zero set is called to hold almost everywhere, or a.e..

\begin{definition}
A measures space $(X, \mathcal{A}, \mu)$ is complete if every subset of a set of measure zero is measureable.
\end{definition}

There is a completion which takes a measure space to a complete measureable space.

\subsubsection{Measure Functions}
A measureable functions is similiar to a continuous functions. A continuous functions pull back open sets to open setes, a measureable functions pulls back measureable sets to measureable sets.


The once we care about are the real-valued functions. We equipped the target $\mathbb{R}$ with the Borel $\sigma$-algebra. 

\begin{definition}
Let $(X, \mathcal{A})$ be a measureable space, then $f : X \rightarrow \mathbb{R}$ if it pulls back Borel sets to to measureable subsets.
\end{definition}

As the Borel $\sigma$-algebra is generated by $(-\infty, b)$, $(-\infty, b]$, $(a, \infty)$, $[a, \infty)$, the pull back of those sets being measureable is suffice for a function to be measureable.

Of course, products, sums, and inverse of a measureable function is measureable. So is $|f|$, $max(f,g)$, $min(f, g)$.

\todo[inline]{ Finish chapter 3 in this subsubsection}



\begin{lemma}

Pointwise limits ($sup, inf, lim sup, lim inf$) of measurable (extended) functions are measureable (extended) real-valued functions. 
\end{lemma}

Therefore if a sequence of measureable functions are pointwise convergent, then its pointwise limit is measureable (as it is lim sup = lim inf).
%%%%%%%%%%%%%%%%%%%%%%%%%%%%

Measureable functions are limits of simple functions:
A characteristic function of $E \subset X$ is a function $\chi_E: X \rightarrow \mathbb{R}$ defined as $\chi_E(x) = 1$ iff $x \in E$ and else $0$. It is measureable iff $E$ is a measureable set.

\begin{definition}
A simple function $\phi: X \rightarrow \mathbb{R}$ is a finite $\mathbb{R}$ linear combination of measureable characteristic functions.
\end{definition}


\begin{theorem}
Any positive measurable function is the pointwise limit of a monotone increasing positive simple functions. 
\end{theorem}

\begin{remark}
In Lebesgue integral we approximate by simple functions , in Riemann integral we partition the domain. So Lebesgue integral we partition the range, in Riemann integral we partition the domain.
\end{remark}

%%%%%%%%%%%%%%%%%%%%%%%%%
In a complete measure space, then all the proposition above we only need almost every property, such as if $f_n \rightarrow f$ almost everywhere (away from a measure-zero set), then $f$ is measureable.


\subsubsection{Integration}

We develope the general theory of integration of a real-valued function on an measure space. Since every measureable function is the pointwise limit of simple functions, we define the simple ones first and extend from there:

For a simple function $\phi = \sum_{i = 1} ^N c_i \chi_{E_i}$, the integral with respect to measure $\mu$ is 
$$
\int \phi d\mu = \sum_{i = 1} ^N c_i \mu(E_i).
$$

\begin{example}
The characteristic function of the rations $\chi_\mathbb{Q}: \mathbb{R} \rightarrow \mathbb{R}$ is not Riemann integrable, but they are Lebesgue integrable with $\int \chi_\mathbb{Q} d\mu = 0$.
\end{example}

For a positive function $f$, then 
$$
\int f d\mu = sup{\int \phi d\mu: 0 \leq \phi \leq f, \phi \ simple}
$$
It is integrable if it is measureable and $\int f d\mu < \infty$


Let $A$ be a measureable set, then 
$$
\int_A f d\mu = \int f_{\chi_A}d\mu
$$

Here's the important Monotone Convergence Theorem:

\begin{theorem}[Monotone Convergence Theorem]
If $\{f_n\}$ is a monotone increasing sequence of positive, measureable, extended real-valued functions, and 
$$
f = lim_{n\rightarrow \infty} f_n,
$$
then 
$$
lim_{n \rightarrow \infty} \int f_n d\mu = \int f d\mu
$$
\end{theorem}

This means that $f$ is the limit of integrals of an increasing sequence of simple functions. This means the following:

\begin{lemma}
In the $L^1$ norm, the space of simple functions are dense. That is, any measureable function is a limit of simple functions in the $L^1$ norm.
\end{lemma}

The measure of an extended real-valued function is its positive part  + the negative part.
It is integrable iff $\int |f| d\mu < \infty$. Note that absolutely convergence of a series is a version of integrable functions.

An important question of integration theory is given $f_n$, does $$
lim_{n \rightarrow \infty} f_n d \mu = \int lim_{n \rightarrow \infty} f_n d \mu
$$.
It is definitely not true in general. 

Here's two important theorems about it:

\begin{theorem}[Fatou's lemma]
Let $f_n$ be sequence of positive measureable functions, then 
$$
\int lim\ inf f_n d\mu \leq lim in \int f_n d\mu
$$
\end{theorem}
%%%%%%%%%%%%%%%%%%%%%%%%%%%%


We also have the dominated convergence theorem:
\begin{theorem}[Dominated Convergence Theorem]
if $f_n$ is a sequence of measureable functions $f_n: X \rightarrow \mathbb{R}$ such that $f_n \rightarrow f$ pointwise, and $|f_n| \leq g$ for $g$ integrable, then
$$
\int f_n d\mu \rightarrow \int f d\mu 
$$
\end{theorem}

Of course, we can generalize all of the above to complex-valued functions.

\begin{theorem}
Of course any Riemann integrable function is Lebesgue integrable and its Riemann integral is equal to Lebesgue intergral. Moreoever, a Lebesgue measureable function is Riemann integrable iff it is bounded and the set of discontinuities has Lebesgue measure zero.
\end{theorem}
\subsection{Lebesgue measure on $\mathbb{R}$}

Any integrable functions on $\mathbb{R}^n$ can be approximated by continuous function with compact support. That is, $L^1(\mathbb{R}^n)$ is the completion of $C_c(\mathbb{R}^n$ with respect to the $L^1(\mathbb{R}^n)$:
\begin{theorem}
The space of compactly supported functions is dense in $L^1(\mathbb{R}^n)$.
\end{theorem}

Let $\mu$ be the Lebesgue Measure on the Real number line $\mathbb{R}$. We want to study how integration on the real number work:

\begin{definition}
Let $f$ be a non-negative function on $\mathbb{R}$, it is Lebesgue measureable if 
\end{definition}





\begin{remark}
The $sigma$ algebra of Lebesgue measureable sets is the completion of the Borel $\sigma$-algebra on $\mathbb{R}^n$. Thus any Borel-measureable function is Lebesgue measureable (inverse is not true!).
\end{remark}






\end{document}