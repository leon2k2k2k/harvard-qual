\documentclass[main.tex]{subfiles}

\newcommand{\F}[1]{
\mathfrak{F}_{#1}}

\begin{document}

\section{Complex Analysis}
Need lots of work here too.
The reference is Complex Analysis by Elias Stein and Rami Shakarchi. Need a lot of residue theorem calculation. They are in Alhfors book. \url{https://www.fing.edu.uy/~cerminar/Complex_Analysis.pdf}

Holomorphic function has many miraculous facts:
\begin{enumerate}
    \item Cauchy's integraction: if $f$ is holomorphic in $\Omega$, then for contractible close path $\gamma$ we have 
    $$
    \int_\gamma f(z) dz = 0
    $$
    \item Regularity: If $f$ is holomorphic, then it is indefinitely differentiable. In fact it is analytic
    \item analytic continuation: If $f$ and $g$ are holomorphic, and equal in arbitrary small disc in $\Omega$, then $f = g$ everywhere.
\end{enumerate}

\subsection{List of Theorems}
\todo[inline]{Cauchy's integration theorem, Cauchy's integration formula, Liouville's theorem, Morera's theorem, Riouche, Open mapping theorem, maximal principal}

\subsection{basics}
\todo[inline]{What is the definition of a holomorphic, meromorphic function, Riemann ?? theorem, Cauchy's integral formula, residues theorem, Gauss's mean value theorem}

First we introduce the notion of holomorphic functions:

\begin{definition}
Let $\Omega$ be an open subset in $\mathbb{C}$ and $f$ complex-valued function on $\Omega$, then $f$ is holomorphic at the point $z_0$ if 
$$
\frac{f(z_0 + h) - f(z_0)}{h}
$$
converges to a limit as $h \rightarrow 0$. The limit, denoted $f'(z_0)$, is the derivative of $f$ at $z_0$. Note that $h$ is complex and can approach $0$ at any direction. $f$ is holomorphic on $\Omega$ if it is holomorphic at every point. If $f$ is holomorphic on $\mathbb{C}$ then we say $f$ is entire. uwu
\end{definition}

\begin{remark}
A holomorphic function is infinitely times complex differentiable and analytic. Much in contrast to real differentiable functions.
\end{remark}

\begin{example} 
Polynomials in $z$ are holomorphic, rational polynomials are holomorphic away from the singularity. $e^z$, $sin \ z$, $cos \ z$ are all holomorphic. $\overline{z}$ is not. 
\end{example}

A function $f$ is holomorphic at $z_0$ iff as a map $\mathbb{R}^2 \rightarrow \mathbb{R}^2$, its derivative exists and the Jacobian matrix is in $\mathbb{C} \subset M_{2 \times 2}\mathbb{R}$.

This means that $f$ satisfies the PDE:
$$
\frac{\partial f}{\partial x} = \frac{1}{i} \frac{\partial f}{\partial y}
$$

Written $f = u + iv$, 
$$
\frac{\partial u}{\partial x} = \frac{\partial v}{\partial y}, \frac{\partial u}{\partial y} = -\frac{\partial v}{\partial x}
$$
These are called the Cauchy-Riemann equations.

We have differential operators
$$
\frac{\partial}{\partial z} = \frac{1}{2} (\frac{\partial}{\partial x} + \frac{1}{i}\frac{\partial}{\partial y})
$$
and 
$$
\frac{\partial}{\partial \overline{z}} = \frac{1}{2} (\frac{\partial}{\partial x} - \frac{1}{i}\frac{\partial}{\partial y})
$$

A function is holomorphic then $\frac{\partial f}{\partial \overline{z}}(z_0) = 0$

\begin{theorem}
$f = u + iv$, if $u$ and $v$ are continuously differentiable and satisfy the Cauchy-Riemann equations. Then $f$ is holomorphic.
\end{theorem}

Given a power series $\sum a_n z^n$, then we can ask where the power series converges.

A series converges absolutely if $\sum|a_n||z|^n $ converges absolutely.

\begin{theorem}
For a power series, there exists $0 \leq R \leq \infty$ such that 
\begin{itemize}
    \item If $|z| < R$ the series converges absolutely.
    \item If $|z| > R$ the series diverges.
\end{itemize}
$R$ is given by the Hadamard's formula 
$$
1/R = lim sup |a_n|^{1/n}
$$
\end{theorem}
uwu
$R$ is called the radius of convergence, $|z| < R$ is the disc of convergence. The boundary $|z| = R$ is more delicate and can converge or divergence. yeet

\begin{theorem}
The poewr series $f(z) = \sum a_n z^n$ is a holomorphic function in its disc of convergence. Its derivative is simply 
$$
f'(z) = \sum n a_n z^{n-1}
$$
with the same radius of convergence.
\end{theorem}

$f$ is analytic at a point $z_0$ if there exists a power series around $z_0$ with positive radius of convergence such that $f(z) = \sum_n a_n (z - z_0)^n$ on an open neighborhood. 




\subsection{Integration}
\begin{remark}
This is my own, more topological and cohomological approach.
\end{remark}
Given a holomorphic function $f(z)$, we can integrate it on a curve $C: [0,1] \rightarrow \mathbb{C}$: $\int_C f(z) dz$. Aka we think about the holomorphic differential $f(z) dz$ and consider its cohomology class:

\begin{lemma}
Any holomorphic differential (differential that looks like $f(z) dz$) is closed. 
\end{lemma}

\begin{proof}
This is because $d = \partial + \overline{\partial}$. Since the top differential form is generated by $dz \wedge d\overline{z} = dx \wedge dy$, and $dz \wedge dz = 0$, we have $d \ f(z) dz = \overline{\partial} f(z) d\overline{z} \d$. However, $\overline{\partial} f(z) = 0$ (this is the definition of $f$ being holomorphic. 
\end{proof}

This means that we have a cohomology class in $H^1(\Omega, \mathbb{C})$, and integration is a homotopic invariant on $C$:

\begin{corollary}[Cauchy's Integration Theorem]
Given $U \subset \mathbb{C}$ and $f$ holomorphic on $U$, then for any closed contractible curve $C$ we have 
$$
\int_C f dz = 0
$$
\end{corollary}

Using this, and a meromorphic function $f$, we can integrate it against a curve bounding singularities of $f$ to get information about the singularity. The standard example is let $f = 1/z$ and $C$ be the unit circle, then 
$$
\int_C 1/z dz = 2\pi i
$$
This is the foundation of the residue theorem. 

\begin{definition}

A primitive for $f$ on $\Omega$ is a $F$ holomorphic on $\Omega$ and $F'(z) = f(z)$. 
That is, $f(z) dz = dF$, or that the holomorphic differential is exact. 
\end{definition}

For example, $1/z$ has no primitive on $\mathbb{C}-0$ because the integration around $0$ is $2\pi i$. It has a antiderivative $log(z)$ at $\mathbb{C} - \mathbb{R}^{>0}$.



By de Rham theorem, we see that 
\begin{theorem}
If $\int_C f dz = 0$ for all $C$, then there exists holomorphic $F$ such that $F' = f$ and $f$ is primitive. Note that this can be check by looking at periods.
\end{theorem}

\begin{theorem}
If $\int_C f dz = 0$ for all $C$, then the cohomology class $[f dz] = 0 \in H^1(\Omega, \mathbb{C})$  by de Rham theorem. This means $f dz$ is exact, thus there is a $F: \Omega \rightarrow \mathbb{C}$ such that $dF = f dz$, as $dF = \partial F dz + \overline{\partial} F d\overline{z}$, we have $\overline{\partial} F = 0$ and $F$ is holomorphic with $F' = f$.
\end{theorem}

This also implies that locally, every holomorphic functions has a primitive. But the gluing of the primitive is obstructed by integration around singularities, aka monodromies.

\subsection{Cauchy's Theorem and Applications}
We want to use Cauchy's theorem to deduce many amazing properties, like the Cauchy's integral formula:

\begin{theorem}[Cauchy's integral formula]
if $f$ is holomorphic on open set containing $C$ and its interior, then for all $z$ in $C$, we have 
$$
f(z) = \frac{1}{2\pi i} \int_C \frac{f(\zeta)}{\zeta - z} d\zeta
$$
\end{theorem}

Differntiation of this identity gives other integral formulas, as well as regularity for holomophic functions. This is remarkable because holomorphicity only assume existence of first derivative, but we get all derivatives. We will also get three consequences:

\begin{enumerate}
    \item Analytic continuation of holomorphic function. 
    \item Liouville's theorem.
    \item Morera's theorem.
\end{enumerate}

Since the real and imaginary part of a holomorphic function is harmonic, it is no surprise that a holomorphic function is determined by boundary values via an integral:

\begin{theorem}
Suppose $f$ is holomorphic on an open set contains the closure of a disk $D$. If $C$ is the boundary circle of this disc with positive orientation, then
$$
f(z) = \frac{1}{2\pi i} \int_C \frac{f(\zeta)}{\zeta - z} d\zeta
$$
for any $z \in D$.
\end{theorem}

\begin{proof}
We can pick a really smaller disk around $z$, then $\frac{f(\zeta)}{\zeta - z}$ has a part with no pole $\frac{f(\zeta) - f(z)}{\zeta - z}$, whose integral around the circle is $0$, and one possibly with pole $\frac{f(z)}{\zeta - z}$, whose integral is $2\pi i f(z)$ (this is the residual theorem).
\end{proof}

Here's a corollary, about the regularity of a holomorphic function as well as integral formulas for its derivatives:

\begin{corollary} [Cauchy integral formulas]
If $f$ is holomorphic in open set $\Omega$, then $f$ has infinitely many complex derivatives. Moreover, for a simple curve $C$ we have 
$$
f^{(n)}(z) = \frac{n!}{2\pi i} \int_C \frac{f(\zeta)}{(\zeta - z)^{(n+1)}} d\zeta
$$
\end{corollary}

We have a bound for the derivative of a holomorphic function by its boundary:
\begin{corollary}[Cauchy's inequality]
If $f$ is holomorphic is an open set that contains the closure of $D$ centered at $z_0$ and of radius $R$, then 
$$
|f^{(n)} (z_0)| \leq \frac{n! \norm{f}_C}{R^n}
$$
where $\norm|f|_C = sup_{z \in C}|f(z)|$ is the supremum of $|f|$ on the boundary circle.
\end{corollary}

\begin{corollary}[Liouville's Theorem]
If $f$ is entire and bounded, then it is constant.
\end{corollary}
\begin{proof}
Cauchy's inequality on $f'$ gives 
$$
|f'| < B/R
$$
where $B$ is a bound of $f$. Take $R \rightarrow \infty$ we get $f' = 0$ thus $f$ is constant. 
\end{proof}

We also have that holmorphic are locally power series:

\begin{theorem}
$f$ is holomorphic on $\Omega$, if $D$ is a disc centered at $z_0$ whose closure is contained in $\Omega$, then $f$ has a power series expansion
$$
f(z) = \sum a_n (a - z_0)^n
$$
with 
$$
a_n = \frac{f^{(n)}(z_0)}{n!}
$$
\end{theorem}

This implies that the power series expansion of $f$ around $z_0$ converges in any disc, as long as its closure is contained in $\Omega$. This also means that if $f$ is entire, then the power series converges for all $\mathbb{C}$.


\subsection{Residue's Theorem, Argument Principal}

The argument principal says that the integral of $f'/f$ over the boundary of a contractible open set counts the zeros and poles of $f$ in the open set.

\begin{theorem}
Let $f(z)$ be a meromorphic function defined on open set $\Omega \subset \mathbb{C}$. Let $C$ be a closed curve in $\Omega$ which avoids all zeros and poles of $f$ and is contractible. Then for each $z \in \Omega$, let $n(C,z)$ be the winding number of $C$ around $z$. Then 
$$
\frac{1}{2\pi i} \int_C \frac{f'(z)}{f(z)} dz = \sum_a n(C,a) - \sum_b n(C,b)
$$
where $a, b$ sums over poles and zeros of $f$, weighted with multiplicities.
\end{theorem}

\begin{proof}
So $f$ is a holomorphic away from the zeros and the poles. So in the homology class $[C] = \sum_a n(C,a) [C_a] + \sum_b n(C, b) [C_b]$. Since $d log(f(z)) =\frac{f'(z)}{f(z)} dz$, $\frac{f'(z)}{f(z)} dz$ belongs in $H^1(\Omega - zeros - poles, \mathbb{C}$. Thus we only have to compute the integral over a small closed loop around each zeros and poles, and expanding $f$ and $f'/f$ near each zeros and poles and the residue theorem proves the statement.
\end{proof}

\begin{remark}
Also 
$$
\int_C \frac{f'(z)}{f(z)} dz = \int_{f \circ C} \frac{dz}{z}
$$
So we are calculating the winding number of $f \circ C$ around $0$, and each zero gives you a positive number equal the multiplicities (it is counterclockwise), and each poles gives you a clockwise rotation.
\end{remark}

From this we have the Rouche's theorem:

\begin{theorem}[Rouche's Theorem]
Given $f(x)$, $g(x)$ holomorphic on a region $K$ with closed contour $C = \partial K$. If $|g(z)| < |f(z)|$ on the boundary $C$, then $f$ and $f + g$ has the same number of zeros inside $K$.
\end{theorem}

\begin{theorem}
The zeros are the winding number of composite around the origin. Note that since $|g(z)| < |f(z)|$, then there is a smooth homotopy between $f(z)$ and $f(z) + g(z)$ avoiding the origin. As the winding number is homotopy-invariant, thus the same the same number of zeros.
\end{theorem}

\begin{remark}
This also gives a proof that $p$ a polynomial of degree $n$ has $n$ roots, as for big enough $R$, then $p(x) = x^n + ... = x^n + (g(x))$. And $|z| = R$ with $f = x^n$ and $g = p(x) - f(x)$ satifies the hypothesis, so $p$ also have $n$ zeros inside $|z| = R$.
\end{remark}

From the analyticity of holomorphic functions, they have only have isolated zeroes without being identically zero. They also have analytic extensions.

Cauchy's integration theorem has an inverse, the Morera's Theorem
\begin{theorem}[Morera's Theorem]
If $f$ is continuous function in open disc $D$, with 
$$
\int_C f(z) dz = 0
$$
then $f$ is holomorphic.
\end{theorem}


\subsection{Meromorphic Functions and Logarithm}

Analytic functions are in an essential way characterized by their singularities. That is, globally analytic functions are determined by their zeros, and meromorphic functions are determined by their zeros and poles. 

There are different levels are singularities:
\begin{enumerate}
    \item removable singularities
    \item poles
    \item essential singularities
\end{enumerate}
Removable ones are harmless. In the third type, function oscillates and grow faster than any power. Second one is more straight forward, easy to understand.

For a holomorphic function $f$ and contractible curve $\gamma$ we have 
$
\int_\gamma f(z) dz = 0
$
What happens when $f$ has a pole? Answer is that it calculates the residues.

The integral (antiderivative) of a holomorphic function with singularities may not be single-valued (the logarithm). Only on simply connected domains can single-valued branches of the logarithm can be defined.

\subsubsection{Zeros and poles}
A point singularity is a complex number $z_0$ such that $f$ is defined in a nbhd of $z_0$ but not at $z_0$. $f(z) = z$ away from $z = 0$ is a removable singularity. $g(z) = 1/z$ has a pole. $h(z) = e^{1/z}$ is essential.

We first by study zeros. 
\begin{theorem}
If $f$ has a zero at $z_0$, not vanish identically. Then exists $U \subset \Omega$ nbhd of $z_0$, a non-vanishing holomorphic function $g$, and unique $n$ such that 
$$
f(z) = (z - z_0)^n g(z)
$$
\end{theorem}

A function $f$ has a pole at $z_0$ if locally $1/f$, with $1/f(z_0) \coloneqq 0$ is holomorphic in a nbhd of $z_0$.

\begin{theorem}
If $f$ has a pole at $z_0$, then exists a nbhd of $z_0$ and a non-vanishing holomorphic function $h$ and unique positive integer $n$ such that 
$$
f(z) = (z-z_0)^{-n} h(z)
$$
\end{theorem}
$n$ is the order. Order $1$ pole are simple.

We have a similar Taylor expansion around a pole:

\begin{theorem}
If $f$ has a pole of order $n$ at $z_0$, then
$$
f(z) = \frac{a_{-n}}{(z-z_0)^n} + ... + \frac{a_{-1}}{z-z_0} + G(z)
$$
where $G$ is a holomorphic function in a nbhd of $z_0$.
\end{theorem}
The sum $\frac{a_{-n}}{(z-z_0)^n} + ... + \frac{a_{-1}}{z-z_0}$ is called the principal part, with $res_{z_0} f \coloneqq a_{-1}$ called the residue. It is special because the other terms, aka the part with order stricly greater than 1, has primitives in a deleted nbhd around $z_0$. Therefore the residue captures the "non-primitiveness" of $f$, more specifically the monodromy.

\subsubsection{The residue formula}
We know consider the residue formula:

\begin{theorem}[Residue formula]
If $f$ is holomorphic away from $z_1, ... z_N$, on open set, $C$ contractible in the open set containing $z_i$ (the winding number of $C$ around $z_i$ is 1). Then 

$$
\int_C f(z) dz = 2\pi i \sum_k res_{z_k} f
$$
\end{theorem}

\subsubsection{Singularities and Meromorphic Functions}

We first give a way to classify removable singularity:

\begin{theorem}
If $f$ is holomorphic in a neighborhood of $z_0$, and bounded in a nbhd of $z_0$, then $z_0$ is a removable singularity. 
\end{theorem}

Also one for poles
\begin{corollary}
If $f$ has an isolated singularity at $z_0$. Then $z_0$ is a pole iff $|f(z)| \rightarrow \infty$ as $z \rightarrow z_0$.
\end{corollary}

So removable singularity, $f$ is locally bounded. For poles, $f$ locally goes to infinity. For essential singularity, it has to be wild: the local image is dense:

\begin{theorem}[Casorati-Weierstrass]
If $f$ is holomorphic at punctured disc 
$D_r(z_0) - z_0$ and has an essential singularity, then the image of the punctured disc under $f$ is dense.
\end{theorem}

We can now define a meromorphic function:
\begin{definition}
A function on $\Omega$ is meromorphic if there exists a sequence of points $z_i$ with no limits such that $f$ is holomorphic away from $z_i$ and has poles at $z_i$. At infinity, if $f$ is holomorphic for all large values of $z$, then consider $F(z) = f(1/z)$, then $f$ has a pole at infinity if $F$ does. If a meromorphic function in the complex plane that is either holomorphic and pole at infinity is said to be meromorphic in the extended complex plane.
\end{definition}

Here's a rigidity theorem:
\begin{theorem}
The meromorphic functions in the extended complex plane are the rationals. 
\end{theorem}
Idea is just to substract out the principal part of each pole.

Note that a rational function is determined up to a constant, by prescribing the locations and the multiplicities of ites zeros and poles.

\subsubsection{Argument principle}
$log f(z)$ can't be globally defined on $\mathbb{C}^\times$. But its differential $f'/f$ can, and for any zeros of order $n$ (might be negative), $f'/f$ has a simple pole with residue $n$. Thus we have that

\begin{theorem}[Argument Principle]
Let $f(z)$ be a meromorphic function defined on open set $\Omega \subset \mathbb{C}$. Let $C$ be a closed curve in $\Omega$ which avoids all zeros and poles of $f$ and is contractible. Then for each $z \in \Omega$, let $n(C,z)$ be the winding number of $C$ around $z$. Then 
$$
\frac{1}{2\pi i} \int_C \frac{f'(z)}{f(z)} dz = \sum_a n(C,a) - \sum_b n(C,b)
$$
where $a, b$ sums over poles and zeros of $f$, weighted with multiplicities.
\end{theorem}

We have the important following theorem 
\begin{theorem}[Open mapping theorem]
$f$ maps open sets to open sets.
\end{theorem}

And a corollary 
\begin{corollary}
If $f$ is non-constant holomorphic function on $\Omega$, then $f$ cannot obtain a maximum in $\Omega$. Its maximum exists on the closure if it is compact.
\end{corollary}

For any simply connected domain, a holomorphic function has a primitive. 

\subsubsection{The complex logarithm}
We want to define $log$ but it is not globally defined as $\theta$ the angle is only defined up to $2\pi$. Do make it defined we need to keep a branch or sheet.

\begin{theorem}
For $\Omega$ simply connected containing $1 \in \Omega$ and $0 \nin Omega$, then exists a branch $F(z) = log(z)$ such that
\begin{enumerate}
    \item $F$ is holomorphic
    \item $e^{F(z)} = z$ for all $z$
    \item $F(r) = log(r)$ for $r$ real and near 1.
\end{enumerate}
\end{theorem}
It is the primitive of $1/z$.

We can also define $log f(z)$ where $f$ does not vanish:

\begin{theorem}
If $f$ is nowhere vanishing holomorphic function in a simply connected region $\Omega$, then exists $g$ such that 
$$
f(z) = e^{g(z)}
$$
\end{theorem}

\subsubsection{Fourier series}
Consider $f$ holomorphic in $D_R(z_0)$ with power series expansion:
$$
f(z) = \sum a_n(z-z_0)^n
$$
The Cauchy integral formula says that
\begin{theorem}
$$
a_n = \frac{1}{2\pi r^n} \int_0 ^{2\pi} f(z_0 + r e^{i\theta} e^{-in\theta} d\theta
$$
for all $n \geq 0$ and $0 < r < R$, and for negative $n$ its zero. 
\end{theorem}

This means that given a holomorphic function, we can consider its restriction on a circle around $z_0$, then its derivatives at $z_0$ are (proportional) to its Fourier coefficients.

Taking the real part, with $f(z_0) = a_0$, we have 
\begin{corollary}
For any $u$ harmonic in a disc $D_R(z_0)$, we have that 
$$
u(z_0) = \frac{1}{2\pi}\int_0 ^{2\pi} u(z_0 + re^{i \theta} d\theta)
$$
for any $0 < r < R$.
\end{corollary}

\subsection{The Fourier Transform}
Given a function on $\mathbb{R}$ with appropriate regularity and decay conditions, we have its Fourier transform. An interesting question is whether $f$ can be extended to a holomorphic function is closely related to rapid decay at infinite of its Fourier transform.

2 ways, one is assme $f$ can be analytically continued in a horizontal strip containing the real axis, and see $\hat{f}$ converges exponentially at infinity. Another way, if $\hat{f}$ is compactly supported, then $f$ has a holomorphic extension with an exponential type growth condition.

\subsubsection{The Class $\F{a}$}
\begin{definition}
Given $a > 0$, $\F{a}$ is class of all function $f$ satisfies
\begin{enumerate}
    \item $f$ holomorphic in horizontal strip $|Im(z)| < a$.
    \item exists $A> 0$ such that 
    $$
    |f(x+iy)| \leq \frac{A}{1 + x^2}
    $$
\end{enumerate}
\end{definition}
$\mathfrak{F}$ is the union of all $\F{a}$. It is closed under derivatives. 

We now consider its Fourier transform:
\begin{theorem}
If $f$ is in $\F{a}$, then $|\hat{f}(\zeta) \leq B e^{-2 \pi b |\zeta|}$
\end{theorem}

Idea is to use Cauchy's integration over a rectangle which is $-R, R, -R-ib, R  ib$ for $g(z) = f(z) e^{-2\pi i z \zeta}$, 
so the integration over the real number line is
the Fourier transform $\hat{f}(\zeta)$, and the two vertical lines are surpressed by the regularity on $f$, and thus we can "lower" the exponential to integrating over the $x -ib $ line, where $g$ is exponentially surpressed by $e^{-2 \pi b \zeta}$.

So $\hat{f}$ has rapid decay at infinity. The more we can extend $f$ (the larget $a$), then larger we can choose $b$, the better the decay. For these function, using the same idea we can prove the Fourier inversion formula and Poisson summation formula.

We can now assume the converse: if $\hat{f}$ satsifes the decay condition $|\hat{f}(\zeta)| \leq A e^{-2\pi a |\zeta|}$, then $f(x)$ is the restriction of a holomorphic function $f(z)$ in the strip $|Im(z)| < b$ for any $0 < b < a$.

Now we have the Paley-Wiener theorem, which gives criterion for when the Fourier transform is supported in an interval $[-M, M]$:
\begin{theorem}
Suppose $f$ is continuous and of moderate decrease on $\mathbb{R}$ (so the Fourier transform exists), then $f$ has an extension to the complex plane that is entire with $|f(z)| \leq Ae^{2 \pi M |z|}$ iff $\hat{f}$ is supported in the interval $[-M, M]$.
\end{theorem}

\subsection{Entire Functions}
Study functions that are holomorphic in the entire whole complex plane.
Questions: When can such functions vanish: given $z_n$ sequence of complex numbers having no limit point, then exists an entire function vanishing exactly at those points. How do they grow at infinity? Jensen's formula. And to what extend are functions determined by their zeros.

\subsubsection{Jensen's Formula}
$D_R$ and $C_R$ open disc and circle of radius $R$ around the origin. $f$ don't vanish identically.

\begin{theorem}
$f$ holomorphic on an open set containing closure of $D_R$, $f(0) \neq 0$ and vanishes nowhere on $C_R$. If $z_1, .. , z_N$ zeros inside the disc counted with multiplicities, then
$$
log |f(0)| = \sum_i ^n log(\frac{|z_k|}{R}) + \frac{1}{2\pi} \int_0 ^{2\pi} log |f (R e^{i\theta} | d\theta
$$
\end{theorem}

This gives a link between the growth of $f$ and its zeros in $D_R$, which we denote $\mathfrak{n}(r)$.

If $f(0) \neq 0$ and $f$ not vanish on circle $C_R$, then
$$
\int_0 ^R \mathfrak{n}(r) \frac{dr}{r} = \frac{1}{2\pi} \int_0 ^{2 \pi} log |f(Re^{i\theta})| d\theta - log|f(0)|
$$

\subsubsection{Functions of Finite Order}
Let $f$ be an entire function, then $f$ has an order of growth $\leq \rho$ if 
$$
|f(z)| \leq A e^{B |z|^{\rho}}
$$
with the order of $f$ 
$$
\rho_f = inf \rho
$$
Growth of $e^{z^2}$ is $2$.

\begin{theorem}
If $f$ is an entire function that has growth $\leq \rho$, then 
\begin{enumerate}
    \item $\mathfrak{n}(r) \leq C r^\rho$ for some $C > 0$ and large enough $r$.
    \item If $z_1, ...$ are the zeros of $f$, with $z_k \neq 0$, then for $s > \rho$ we have 
    $$
    \sum_k \frac{1}{|z_k|^s} \leq \infty
    $$
\end{enumerate}
\end{theorem}

$f(z) = sin \pi z$ shows that the convergence in part ii is sharp. 

Now we study given $z_i$ countable, can we find $f$ entire such that it vanishes precisely at $z_i$. This uses infinite product, but the naive product $\Pi(z - z_i)$ is not convergent.

\subsubsection{Infinite Products}

\begin{proposition}
If $\{F_n\}$ is sequence of holomorphic functions on $\Omega$, if exists $c_n > 0$ such that 
$\sum c_n < \infty$, and $|F_n(z) - 1| \leq c_n$, then 
\begin{enumerate}
    \item the product $\prod F_n(z)$ converges uniformly in $\Omega$ to a holomorphic function $F(z)$.
    \item If $F_n(z)$ doesn't vanish, then
    $$
    \frac{F'(z)}{F(z)} = \sum     \frac{F'_n(z)}{F_n(z)}
    $$
\end{enumerate}
\end{proposition}

An example is 
$$
\frac{sin \pi z}{\pi } = z \prod (1 - \frac{z^2}{n^2}
$$

Here's the theorem:
\begin{theorem}[Weierestrass infinite products]
Given $a_n$ with $|a_n| \rightarrow \infty$, then there exists an entire function $f$ such that $f$ vanishes at $z = a_n$ and nowhere else. Any other such entire function is of the form 
$$
f(z) e^{g(z)}
$$
where $g$ is entire.
\end{theorem}


\subsection{The Gamma and Zeta Functions}
They are super important, especially Zeta function.

\subsubsection{Gamma Function}
For $s > 0$, the gamma function is 
$$
\Gamma(s) = \int_0 ^{\infty} e^{-t} t^{s-1} dt
$$

The integral is convergent and extends to an analytic function for half-plane $Re(s) > 0$.

For $Re(s) \leq 0$, then the integral is not absolutely convergent and we need to analytic continue it.

We have the following property:
\begin{lemma}
If $Re(s) > 0$, then 
$$
\Gamma(s+ 1) = s \Gamma(s)
$$
so $\Gamma(n+1) = n!$
\end{lemma}

\begin{theorem}
The function $\Gamma(s)$ initially defined for $Re(s) > 0$ has an analytic continuation to a meromorphic function on $\mathbb{C}$ whose only ingularities are simple poles at negative integers $s = 0, -1, ... $, then residue of $\Gamma$ at $s = -n$ is $(-1)^n /n!$.
\end{theorem}
The idea is that we simply extend by the identity $\Gamma(s+ 1) = s \Gamma(s)$ to $Re(s) > -1$. By the identity above, for $Re(s) > 0$ it agrees with the old $\Gamma$. Thus this is the analytic continuation.

$\Gamma$ is symmetric about the line $Re(s) = 1/2$:

\begin{theorem}
$$
\Gamma(s) \Gamma(1-s) = \frac{\pi}{sin \pi s}
$$
\end{theorem}
Because both sides are holomorphic, we simply have to check in the strip $0 < Re(s) < 1$.

We see that 
\begin{theorem}
$\frac{1}{\Gamma(s)}$ is an entire function with simple zeores at $s = 0, -1, ...$. It has growth order $1$.
\end{theorem}

Thus we have the product formula:

\begin{theorem}
For all $s \in \mathbb{C}$, 
$$
\frac{1}{\Gamma(s)} = e^{\gamma s} s 
$$
\end{theorem}

$$\gamma = \lim \sum \frac{1}{n} - log N$$

This means that the function $\frac{1}{\gamma}$ is essentially characterized as the entire function that has simple zeros at $s = 0, -1, -2...$ and vanishes nowhere else, and its order of growth $\leq 1$.


\subsubsection{The Zeta Functions}

The Riemann zeta function is defined for $Re|s| > 1$ by 
$$
\zeta(s) = \sum \frac{1}{n^s}
$$

It is uniform as it converges uniformly on any $Re(s) > 1 + \delta + 1$.

To analytically extend it, consider the theta function 
$$
v(t) = \sum_{-\infty} ^{\infty} e^{\pi n^2 t}
$$
Poisson summation formula gives a functional equation 
$$
v(t) = t^{-1/2} v(1/t)
$$

Let 
$$
\xi(s) = \pi^{-s/2}\Gamma(s/2)\zeta(s).
$$
Then 
\begin{theorem}
The function $\xi$ is holomorphic for $Re(s) > 1$ and has an analytic continuation to all of $\mathbb{C}$ as a meromorphic function with simple poles at $s = 0$ and $s = 1$, with 
$$
\xi(s) = \xi(1-s)
$$
\end{theorem}

Thus we have 
\begin{theorem}
The zeta function has a meromoirphic continuation into the entire complex plane, whose only singularity is a simple pole at $s = 1$.
\end{theorem}
%%%%%%%%%%%%%%%%%%%%%%%%%%%
\subsection{Rigidity of holomorphic functions}

We have the maximum modulus principal:

If $f$ is a holomorphic function, the $|f|$ can't have a local maximum within the domain of $f$:

\begin{theorem}
Let $f$ be a holomorphic function on a connected oepn subset $D$ in $\mathbb{C}$. Then $z_0$ is a point in $D$ such that 
$$
|f(z_0)| \geq |f(z)|
$$
then $f$ is holomorphic.
\end{theorem}
This also means that 

\begin{corollary}
The maximum of a holomorphic function inside of $\overline{D}$ is on its boundary.
\end{corollary}
This also follows from the open mapping theorem, which says that any nonconstant holomorphic function maps open sets to open sets.

Holomorphic function are incredibly rigid. For example, they are not only smooth but also analytic. By Cauchy's integral formula, we have the following:

\begin{theorem}[Liouville's Theorem]
Any bounded entire (defined on the whole $\mathbb{C}$ function is constant
\end{theorem}

We can prove the fundamental theorem of algebra easily from Liousville's theorem:

\begin{corollary}[Fundamental Theorem of Algebra]
Any nonconstant polynomial $p(x)$ has a root in $\mathbb{C}$
\end{corollary}
\begin{proof}
$p(z)$ goes to infinity as $z \rightarrow \infty$. So consider $1/p(z)$, it is less than $1$ for $|z| > R$, and $p(z)$ has a  maximum bound for $|z| < R$. Thus $1/p(z)$ is an entire bounded function, thus it has constant by Liouville's theorem. Contradiction.
\end{proof}

We also have that any positive harmonic function is constant:

\begin{corollary}
Any positive harmonic function $u$ is constant
\end{corollary}

\begin{proof}
We have an entire holomorphic function $f$ such that the positive part of $f$ is $u$. Then the 
\end{proof}
\subsection{Holomorphic and Harmonic Functions}

The Laplancian differential operator 
$$
\Delta = \frac{\partial^2}{\partial x^2} + \frac{\partial^2}{\partial y^2} = 4 \frac{\partial}{\partial \overline{z}} \frac{\partial}{\partial {z}}
$$

The real and imaginary part of a holomorphic function $f$ is harmonic.

\todo[inline]{the relationship between harmonic and holomorphic functions.}
A real-valued function on the $\mathbb{C}$ plane is harmonic if 

\end{document}