\documentclass[main.tex]{subfiles}
\usepackage[utf8]{inputenc}

\title{DG}
\date{July 2021}

\begin{document}

\maketitle


%%%%%%%%%%%%%%%%%%%%
\section{Overview}
There is a lot of equation and stuff to remember for DG. A lot of connection stuff, but I think (thankfully) no gauge theory.

%%%%%%%%%%%%%%%%%%%%
\section{Connection and curvature}


\begin{definition}
A connection $\nabla$ on a vector bundle $E$ on a manifold $M$ is a linear map 
$$
\nabla: \Gamma(M, E) \rightarrow \Gamma(M, \Omega^1(M) \otimes E )
$$
satisfying the Leibniz rule:
$$
\nabla(f s) = f \nabla(s) + df \otimes s
$$
where $f \in C^{\infty}M$ and $s \in \Gamma(M)$. Given $X \in \Gamma(M)$, we can define 
$\nabla_X(s) \coloneqq \nabla(s) (\mu)$. Note that this is $C^{\infty}M$-linear in $X$, that is, $\nabla_{fX}(s) = f \nabla_{X}(s)$
\end{definition}

Naturally, if I have a connection $\nabla$ on $p: E \rightarrow M$, and a map $f : N \rightarrow M$, then we get pullback connection $f^*\nable$ on $f^*E \rightarrow N$.

In local coordinates, we have basis $s_j$ for $E$ and $dx^i$ for $T^*M$, then we have 
$$
\nabla(s_j) = \Gamma_{ij}^k dx^i \otimes s_k
$$
Equivalently, 
$$
\nabla_{\partial_i} s_j = \Gamma_{ij}^k s_k
$$
$\Gamma_{ij}^k$ are called the Christoffel symbols.

Alternatively, one can write
$$
\omega(s_j) = \omega_j ^k s_k
$$
where $\omega_j ^k = \Gamma_{ij}^k dx^i$ are connection 1-forms.

From this and Leibnitz rule we can determine the formula for any local section.
Given a covariant derivative $\nabla$ on $E$, we can extend $d_\nabla: \Gamma(M, \Omega^i(M) \otimes E) \rightarrow \Gamma(M, \Omega^{i+1}(M) \otimes E)$. 

We also get connection on tensors, direct sums, and duals of a vector bundle:
\begin{construction}
Given $(E, \nabla')$, $(F, \nabla'')$, we get a connection $\nabla$ on $E \sum F$ as follows:
$$
\nabla(s, t) = (\nabla' s, \nabla'' t)
$$

Similarly, we have connection $\nabla$ on $E \otimes F$:
$$
\nabla s\otimes t = \nabla's \otimes t + s \otimes \nabla'' t
$$

The formula on the symmetric and exterior powers are the same.

lastly, there is a connection on the dual, defined by:
$$
\nabla \alpha (s) \coloneqq d (\alpha(s)) -  \alpha (\nabla' s) 
$$
This is uniquely determined by making the pairing map $(E \otimes E^*, \nabla' \otimes \nabla) \rightarrow (\underline{\mathbb{R}}, d)$ a map of bundles with connections (that is, taking connection commutes).
\end{construction}


%%%%%%%%%%%%%%
\subsection{Curvature}

\todo[inline]{Mean curvature, scalar curvature etc, Ricci curvature, the symmetry of the Riemann curvature tensor}
From this, we can define the curvature of a connection:

\begin{definition}
Given a connection $\nabla$ on $E$, we have the curvature:
$$
R(\mu, \nu) s  \coloneqq \nabla_{\mu} \nabla_{\nu} s - \nabla_{\nu} \nabla_{\mu} s  - \nabla_{[\mu, \nu]} s
$$
In fact, $R: \Gamma(E) \rightarrow \Gamma(\Omega^2(M) \otimes E)$ is precisely $d_\nabla ^2$. It is in fact a tensor (that is it is linear in $\mu, \nu$ and $s$).

In coordinates, 
$$
R^l _{k, i, j} = dx^l (R(\partial_i, \partial_j) \partial_k) = \partial_i \Gamma_{jk}^l - \partial_j \Gamma_{ik}^l + \Gamma_{im}^l \ \Gamma_{jk}^m   -  \Gamma_{jm}^l \  \Gamma_{ik}^m 
$$

Using connection 1-forms $\omega$, we have 
$$
R \coloneqq d\omega + \omega \wedge \omega
$$
Note $\omega \wedge \omega$ does matrix multiplication on the $(k,l)$ components. 

A connection is flat if its curvature is 0. Thus gives local systems on the manifold.
\end{definition}


\subsection{Torsion}
The most interesting connections are the ones on the tangent bundle, which is what we called a connection on a manifold.

\begin{definition}Given a connection $\nabla$ on $M$ ($TM$), then its torsion tensor is 
$$
T(X,Y) \coloneqq \nabla_X(Y) - \nabla_Y(X) - [X,Y]
$$
It can be shown that this is a tensor ($C^{\infty}M$-linear) and ofc anti-symmetric.
\end{definition}

In local coordinates and Christoffel symbols, this means that 
$$
\Gamma_{ij}^k = \Gamma_{ji}^k
$$
%%%%%%%%%%%%%%%%%%%%%%%
\subsection{Parallel Transport}
Given a bundle $E$ over the interval $[0,1] \in \mathbb{R}$, together with a connection $\nabla$. At any point $v \in E|_0$, there exists a unique section $s_v : [0,1] \rightarrow E$ with starting point $v$ such that $\nabla_{\partial_x} s_v = 0$.

Locally, write a basis for $E$ with $s_v$ being an basis element (ofc, when $s_v$ is nonzero), and we have Christoffel symbol:
$$
\nabla_{\partial_x} s_i = \Gamma_{i}^k s_k
$$
By the fundamental theorem of ODE, we see that this has a unique solution that exists for all $x \in [0,1]$.

Thus given a connection $\nabla$ on $E \rightarrow M$, and a path $\gamma: [0,1] \rightarrow M$, then we get a parallel transport map: $E|_{\gamma(0)} \rightarrow E|_{\gamma(1)} $. Ofc this is compatible with concatenation. Moreover, the difference between how infinitesimally small path give different parallel transport is precisely measure by the curvature. So when the curvature vanishes, then any two homotopic smooth paths gives the same transport, and we get a local system. Another way to put it is that we have local flat sections, over a local neighborhood, not just a path.

%%%%%%%%%%%%%%%%%%%%%%%
\subsection{Geodesics}
Given a connection $\nabla$ on (the tangent space of) manifold $M$. Given any path $\gamma: [0,1] \rightarrow M$, we can ask for the following
$$
\nabla_{\dot{\gamma}} \dot{\gamma} = 0
$$
In coordinates, where $t$ is the coordinate of the interval, and $x^i$ are the local coordinates of the manifold, we have the geodesic equations:
$$
\frac{d^2 x^k}{dt ^2} + \Gamma_{ij}^k \frac{dx^i}{dt} \frac{dx^j}{dt} = 0
$$
Note that this is $n$ 2nd order ODE in $n$ variables, and the existence and uniqueness theorem tells us that there exists locally a unique solution, determined by the starting point as well as the velocity. That is, given any $p \in M$ and $v \in T_p M$, there is locally a unique geodesic going through $p$ with velocity $v$.
%%%%%%%%%%%%%%%%%%%%
\section{Riemannian Geometry: General Theory}
Let $(M, g)$ be a Riemannian manifold, then we have an inner product 
$(-,-)$ on $T_p M$ for every point $p \in M$. This identifies vectors with covectors, and gives us a way to turn $(i, j)$ tensors to $(i-1, j+1)$ tensors and vice versa. This is the raising and lowering of indicies.

Given a Riemnnian metric $g$ on a vector bundle $E$, a connection $\nabla$ on $E$ is compatible with the metric if $\nabla g = 0$. In formula, this means that for any $X \in TM$, $s, t$ sections of $E$, we have 
$$
d(s,t) = (\nabla s, t) + (s, \nabla t)
$$

\begin{definition}
There is a unique torsion-free, compatible connection $\nabla$ associated to $g$, called the Levi-Civita connection.
\end{definition}
In local coordinates, 
$$
\Gamma_{ij}^k = \frac{1}{2} g^{kl}(\partial_i g_{jl} + \partial_j g_{il} - \partial_l g_{ij})
$$
\begin{remark}
Given a 1-form $\alpha$, its covariant derivative $\nabla \alpha$ is the dual of $\nabla V_\alpha$, where $V_\alpha$ is the dual of $\alpha$.
\end{remark}

\begin{remark}
On a Riemannian manifold $M$, the holonomy around a loop $C$ is the parallel transport of the tangent vector around the loop.
\end{remark}

\subsection{Curvature of Riemannian manifolds}
Recall that we have the curvature tensor $R_{ijk}^l$. We can also lower the $l$ to get the tensor $R_{ijkl}$. This tensor has a lot of symmetries:
$$
R_{ijkl} = R_{klij}, R_{ijkl} = -R_{jikl} (R_{ijkl} = - R_{ijlk}), R_{ijkl} + R_{iljk} + R_{iljk} = 0
$$
In dimension $n$, it has 
$$
\frac{n^2 (n^2 - 1)}{12}
$$
components.
Taking trace gives us the Ricci curvature:
\begin{definition}
The Ricci curvature tensor is 
$$
R_{jk} = R^i _{kij}
$$
It is symmetric in j and k. 
\end{definition}

Taking trace once more, we get the scalar curvature
$S = R^j _j = g^{ij} R_{ij}$.

%%%%%%%%%%%%%%%%%%%%
\subsection{Riemannian Geometry: Submanifolds}

Given an immersion of manifold $i: N \rightarrow M$, then a Riemannian metric $g$ on $M$ pullback to a Riemannian metric $g'$ on $N$. However, we can define the Levi-Civita connection
$\nabla'$ on $N$ more easily:
\begin{lemma}
$\nabla'_{\mu} \nu = pr (\nabla_{\mu} \nu)$, where the projection is $TM \rightarrow TN$ using the metric $g$ on $M$. 
\end{lemma}

We can also project down to the normal bundle:
\begin{definition}
$II: TN \times TN \rightarrow N$ is defined as $II(X, Y) \coloneqq (\nabla_{X} Y)^\perp$, where $\perp$ is the projection $TM \rightarrow N$, where $N$ is the normal bundle. This is symmetric because the Levi-Civita connection is torsion-free (and the lie bracket of two tangent vector field that lies on a submanifold stays on the submanifold).
Given $\mathfrak{n} \in N$, then we get a pairing $II_\mathfrak{n}: TN \times TN \rightarrow \mathbb{R}$ given by 
$$
X, Y \mapsto (\nabla_{X} Y, \mathfrak{n}) = - (\nabla_{X} \mathfrak{n}, Y)
$$
Note that the equation is by the compatibility of $\nabla$ with $g$ as well as the fact that $(Y, \mathfrak{n}) = 0$ as $\mathfrak{n}$ is normal.

The compatibility conditions implies that 
$\nabla_X v$ is perpedicular to $v$ when $v$ is unit lenghth. Thus the holonomy are in $SO(T_pM)$. 

Here's some equivalent definition for $\nabla$ to be compatible with $g$:
\begin{enumerate}
    \item $\nabla g = 0$.
    \item the pairing $(-,-): TM \times TM \rightarrow \underline{\mathbb{R}}$ is map of bundles with connections.
    \item the iso $TM \iso T^*M$ arising from $g$ is a map of bundles with connections.
\end{enumerate}

Note the last one tells us how to compute covariant derivative for 1-forms: just dualize.

Dually, we have the shape operator associated to $\mathfrak{n}$: 
\begin{align*}
S_{\mathfrak{n}}: TN &\rightarrow TN \\ 
X  \mapsto & - (\nabla_{X} \mathfrak{n}) 
\end{align*}
\end{definition}

In coordinates, we have 
$$
S^i _j =  II^{ik} g_{kj}
$$
One can check that $\nabla_{X} \mathfrak{n}$ lies on the tangent space of $M$

As the shape operator is symmetric (with respect to the Riemannian inner product of $N$), its eigenvalues are all real, and they are all orthogonal, no generalize eigenspaces. Thus there are eigenvalues and eigensubspaces of $S$. The eigenvalues are called the principal curvatures and the tangent vector there curve in circle around the normal vector.

\subsection{Geodesics in a Riemannian manifold}
Given a Riemann manifold $(M, g)$, since it has a Levi-Civita connection, thus we can ask for geodesics. But in fact there is a more direct interpretation (and use more often for quals), which is using the Lagrangian formulism.

Given any path $\gamma: [0,1] \rightarrow M$, we have the following functional 
$$
\mathcal{L} \coloneqq \int_{[0,1]} g_{ab} \dot{\gamma^a} \dot{\gamma b}
$$, this is a form of length (or energy) of the path.

whose Euler Lagrange equation is precisely the geodesic equation. 

Recall the Euler-Lagrange equation, whose solution are the local extrema of the variation problem is given by 
$$
\frac{\partial \mathcal{L}}{\partial x^i} - \frac{d}{dt} \frac{\partial \mathcal{L}}{\partial \dot{x^i}} = 0
$$

\subsection{Covariant derivative in Orthogonal frames}

On a Riemannian manifold $(M, g)$, given an orthogonal frame $\theta_i$ and coframe $\theta^i$, we would like to calculate the covariant derivative (Christoffel symbols) of the Levi-Civita connection $\nabla$ from the frame. Using torsion-free and compatibility property.

\todo[inline]{Orthonormal frame and Cartan's structure equation, also using the connection 1-form $\omega$}

In this frame, we have the connection 1-form 
$$
\nabla \theta_j = \omega_j ^k \theta_k
$$
with 
$$
\omega_j ^k = \Gamma_{ij}^k \theta^i
$$
As $(s_i, s_j) = \delta_{i,j}$, the compatibility condition says that 
$$
(\nabla \theta_j, \theta_k) = -(\nabla \theta_k, \theta_j)
$$
On coordinates, this means that 
$$
\omega_j ^k = - \omega_k ^j
$$

Using the torsion free assumption, we also have the 1st Cartan's structural formula:
$$
d\theta^k = \omega_i ^k \wedge \theta^i
$$
This formula is super important in calculating covariant derivatives. This is a part of the orthonormal decomposition method!
(evaluating both sides at $(\theta_m, \theta_l)$, then the left hand side is given by $-\theta^k([\theta_m, \theta_l])$ while the right hand side gives the corresponding $k$-th Christoffel symbol for $\nabla_{\theta_m} \theta_l - \nabla_{\theta_m}\theta_l \ d\theta^k= \omega^k _{ml} - \omega^k _{lm}$.

\section{Riemannian Geometry: Surfaces in $\mathbb{R}^3$}
\todo[inline]{Gauss map, interpreting the curvature from the Gauss map, Gauss-Bonnet theorem.}

Let $S \subset \mathbb{R}^3$ be an embedded smooth surface, and $\mathfrak{n}$ a normal vector (this implicitly assumed a co-orientation of $S$, equivalently, an orientation of $S$, since $\mathbb{R}^3$ is oriented).

Since there is a canonical choice of normal vector, we get the second fundamental form. At any point $p$, the shape has two (not necessarily distinct) eigenvalues, $k1$ and $k2$, and they are the maximal point of the curvature of the normal planes intersecting with the curve. The Gaussian curvature $K \coloneqq k_1 k_2$. It is the determinant of the shape operator, equivalently, in any coordinate, 
$$
K = det S = \frac{det II}{det I}
$$

The mean curvature is $k_1 + k_2$ and it is the trace of the shape operator:
$$
H = tr(S) = g_{ij} II^{ij}
$$

For the Euclidean space $\mathbb{R}^n$ with its standard coordinates $x^i$, the covariant derivative is simply taking derivative, that is, all the Christoffel symbol are all zero.

In three dimensions, there are vector calculus calculations we can use, namely the cross product. Given a parametrized surface $\Phi: U \rightarrow \mathbb{R}^3$, to find the normal vector, we take two independent local vector field $\mu$, $\nu$ on $U$, then we $\Phi_* {\mu}$ and $\Phi_* \nu$ spans the tangent spaces of the surfaces, and the cross product $\Phi_* \mu \times \Phi_* \nu$ is perpendicular to both, those spans the normal line bundle. Therefore 
$$
\mathfrak{n} \coloneqq \frac{\Phi_* \mu \times \Phi_* \nu}{|\Phi_* \mu \times \Phi_* \nu|}
$$
is a unit length normal vector. 

\subsection{Curvature in 2 space}

In two dimensions, the Riemann curvature tensor only have 1 degree of freedom and is determined by 
$R^1 _{212}$. 

The relationship between Riemann curvature and Gaussian curvature is as follows:
$$
R_{1212} = det(g) K
$$
Thus in with orthonormal frames they are the same. Coordinate independently, we have 
$$
R_{klij} = K (g_{ki}g_{lj} - g_{kj}g_{lj})
$$
and multiple with $g^{ki}$, we get 
$$
Ric_{lj} = K g_{lj}
$$
Multiple with $g^{lj}$, we get that the scalar curvature:
$$
S = 2K
$$

\begin{remark}
Note that holonomies of loops on a surface are rotations as $SO(2) = U(1)$.
\end{remark}
\subsection{Gauss-Bonnet, Theorem-Egrenium}
\todo[inline]{State those theorem, at least know the proof of Theorem-Egrenium, and how does Guass-Bonnet relate to the first chern class of the tangent bundle?}

Given a surface $S \subset \mathbb{R}^3$, Theorem Egregium relates the Gaussian curvature, an external curvature invariant, with the Riemann curvature tensor coefficients (or the scalar curvature). It says that 
$$
K = R(X, Y) X Y
$$
for $X, Y$ local orthonormal vector fields. 

On the other hand, the Gauss-Bonnet theorem says that the integral of the Gaussian curvature is a topological invariant:
$$
\int_S K dA = 2 \pi \chi(M)
$$

It has an extension to compact Riemannian manifold $S$ with boundary:
$$
\int_S K dA + \int_{\partial S}k_g ds = 2\pi \chi(M)
$$
where $k_g$ is the geodesic curvature of the boundary of $S$ (as a curve in $S$).

Given a curve $\gamma$, then 
$$
\delta_\dot{\gamma} \dot{\gamma}
$$
is the normal vector of the curve. In dimension 2, this is a multiple of the normal vector, thus we get a scalar $k_g$.

\subsection{Orthonormal method to compute curvature of surfaces}
This is a standard method for computing curvature in 2-dimensions. First we take a orthonormal coframe $\theta^i$ and the dual frame $\theta_i$. We have the connection 2-form $\omega_j ^k$. From above, we have that 
$$
\omega_j ^k = - \omega_k ^j
$$
This tells us right away that we only have to calculate 
$$
\omega_1 ^2 = - \omega_2 ^1
$$
Then using Cartan's structural equation we have $$
d\theta^k = \omega^k _i \wedge \theta^i
$$
and this gives us two equations
$$
d\theta^1 = \omega^1 _2 \wedge \theta^i
$$ and 
$$
d\theta^2 = \omega^2 _1 \wedge \theta^1
$$. 
Hopefully this is enough to compute $\omega^1_2$. Then we have 
$$
R = d \omega + \omega \wedge \omega
$$


\section{Calculus of differential forms}
Lie brackets and exterior derivatives are in some sense dual.

The Lie bracket has the unique properties  that 
\begin{enumerate}
    \item $\mathcal{L}_X \sigma \otimes \tau = \mathcal{L}_X \sigma \otimes \tau + \sigma \otimes \mathcal{L}_X \tau$
    \item $\mathcal{L}_X f = X f = df (X)$
    \item $\mathcal{L}_X \sigma(Y_1, ..., Y_n) = \mathcal{L_X} (\sigma)(Y_1, ..., Y_n) + \sigma (\mathcal{L_X}Y_1, Y_2,...) + ...$
\end{enumerate}

For a 0-form $f$, we have 
$$
df(X) = X(f) = \mathcal{L}_X f
$$

For a 1-form $\omega$, we have 
$$
d\omega(X,Y) = X(\omega(Y)) - Y(\omega(X)) - \omega([X,Y])
$$

We have the Cartan's magic formula:
$$
\mathcal{L}_X \omega = (\iota_X d + d \iota_X) \omega
$$

There is a formula expressing $d\omega$ as evaluation of $\omega$ and lie brackets.
Using the Cartan's magic formula as well as expanding $\mathcal{L}_X$ on pairings of vectors and covectors.

\section{Standard Coordinates}
The polar coordinates on $\mathbb{R}^2 - {0}$:
$$
x = r cos \theta, y = r sin \theta
$$
. The standard metric becomes 
$$
ds^2 = dr^2 + r^2 d\theta^2
$$
with $r \in \mathbb{R}^{>0}$ and $\theta \in [0,2\pi)$
The spherical coordinate on $S^2$ is 
$$
x = cos \phi \ sin \theta, 
y = sin \phi \ sin \theta, 
z = cos \theta 
$$
with $\phi \in [0,2\pi)$ and $\theta \in (0, \pi)$
the sphere metric becomes 
$$
ds^2 = d\theta^2 + sin^2 \theta \ d\phi^2
$$
\end{document}
