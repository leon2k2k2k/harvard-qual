\documentclass[main.tex]{subfiles}
\usepackage[utf8]{inputenc}

\title{DG}
\date{July 2021}

\begin{document}

\maketitle


%%%%%%%%%%%%%%%%%%%%
\section{Overview}
For algebra, the problem seems to fall into three categories: group theory, rep theory, and ring/Galois theory.

\section{Group Theory}
the most important thing here is the Sylow theorems and how they are used. Sylow's theorem is about subgroups of prime orders.

Here's a good link: \url{http://math.uchicago.edu/~may/REU2016/REUPapers/Idelhaj.pdf}

\begin{theorem}[Sylow's theorem]
Let $G$ be a fintie group and $p$ a prime number. If $p^n$ divides the order of $G$, then $G$ has a subgroup of order $p^n$.
\end{theorem}

We also have the Cauchy's theorem:

\begin{theorem}[Cauchy's theorem]
If prime $q$ divides $|G|$, then there exists an element of order $q$.
\end{theorem}

Let $G$ acts on $G$ by conjugation, then the orbits are conjugacy classes. 
Then the class equations says that $|G| = |Z(G)| + k_1 + k_2 ..$, where $|Z(G)|$ is the center of $G$ (those that commutes with all elements those belong to a single orbit), and $k_1$ are the size of distinct non-singlet orbits.


\subsection{Groups of order $p^2$}
Groups of order $p$ are isomorphic to $\mathbb{Z}/p\mathbb{Z}$, namely abelian. Similar thing is true for groups of order $p^2$:

\begin{theorem}

\end{theorem}

\subsection{Groups of order $pq$}

\subsection{Groups of order $p^3$}



\subsection{Dihedral Groups}
\todo[inline]{Their generator and relations, as well as their conjugacy classes and representations}



\section{Rep Theory}
This concerns $\mathbb{C}$-linear representation theory of finite groups, via the character theory.


\subsection{Characters table for specific groups}



\section{Galois Theory}
\todo[inline]{Review the basics of field extensions and Galois theory, discriminants. How to know if a Galois extension is in $A_n$ etc. Examples of basic construction of Galois extensions that are like $\mathbb{Z}/2\mathbb{Z} \tiems \mathbb{Z}/2 \mathbb{Z}$ etc.}



\section{Ring Theory and Algebraic Number Theory}
\todo[inline]{Write about integrally closed, UFD etc. Prime factorization in a Dedekind domain, quadratic extensions $\mathbb{Z}[\sqrt{n}]$ and rather is it is integrally closed/UFD.}

\end{document}